% ****** Start of file templateForReport.tex ******

% TeX'ing this file requires that you have all prerequisites
% for REVTeX 4.2 installed
%
% See the REVTeX 4 README file
% It also requires running BibTeX. The commands are as follows:
%
%  1)  latex templateForReport.tex
%  2)  bibtex templateForReport
%  3)  latex templateForReport.tex
%  4)  latex templateForReport.tex
%
\documentclass[%
 reprint,
%superscriptaddress,
%groupedaddress,
%unsortedaddress,
%runinaddress,
%frontmatterverbose,
%preprint,
%showpacs,preprintnumbers,
%nofootinbib,
%nobibnotes,
%bibnotes,
 amsmath,amssymb,
 aps,
%pra,
%prb,
%rmp,
%prstab,
%prstper,
%floatfix,
]{revtex4-2}

\usepackage{graphicx}% Include figure files
\usepackage{dcolumn}% Align table columns on decimal point
\usepackage{bm}% bold math
\usepackage{todonotes}

%\usepackage{hyperref}% add hypertext capabilities
%\usepackage[mathlines]{lineno}% Enable numbering of text and display math
%\linenumbers\relax % Commence numbering lines

%\usepackage[showframe,%Uncomment any one of the following lines to test
%%scale=0.7, marginratio={1:1, 2:3}, ignoreall,% default settings
%%text={7in,10in},centering,
%%margin=1.5in,
%%total={6.5in,8.75in}, top=1.2in, left=0.9in, includefoot,
%%height=10in,a5paper,hmargin={3cm,0.8in},
%]{geometry}

\begin{document}

\title{Advanced Methods of Data Analysis: Normalizing Flows}% Force line breaks with \\
%\thanks{A footnote to the article title}%

\author{Leonhard Moske}
%\author{Second Author}
 %\email{Second.Author@institution.edu}

%\author{Charlie Author}
% \homepage{http://www.Second.institution.edu/~Charlie.Author}

%\author{Delta Author}

\date{\today}% It is always \today, today,
             %  but any date may be explicitly specified

\begin{abstract}
%  An article usually includes an abstract, a concise summary of the work
%  covered at length in the main body of the article.
%  \begin{description}
%  \item[Usage]
%    Secondary publications and information retrieval purposes.
%  \item[Structure]
%    You may use the \texttt{description} environment to structure your abstract;
%    use the optional argument of the \verb+\item+ command to give the category of each item.
%  \end{description}

In this paper, the expressiveness of fully connected neural networks is used in a class of transformation known as normalizing flows in order to construct density estimator. Further these estimators are utilized to classify stars\todo{more explizit}.

\end{abstract}
\maketitle

%\tableofcontents

\section{Introduction}
\todo{what do we do? -> classify by density estimation}
Normalizing flows is a powerful method that utilizes the transformation of random variables for either density estimation or for generative sampling.

what is data -> features of stars here difference of insenity with different filters.


\section{Theory}
To classify data into different categories using the density of these categories we have to evaluate the probability density distributions at the data and compare the result, i.e. choose the category with the highest probability. In this case we have only two categories (background named 0, signal named 1), thus we can use a test statistic like:
\begin{align*}
	t(\text{data}) = \ln(\frac{p(0|\text{data})}{p(1|\text{data})})
\end{align*}
We should classify the data as background if $t>0$ and as signal if $t<0$. In practice this cut has to be chosen with respect to the validation of the classifiers. 

So we have to construct an algorithm that allows us to evaluate the probability density of both categories. 
\todo{classifier by density estimation: if we have density we can estimate p(signal|background) -> compare all likelyhoods <- check this term}
\subsection{functions of random variables}
To estimate the density distribution we make use of the formula of  transformations of random variables, which lets us connect a simple distribution that we can evaluate and the distribution of the data. This allows us to approximate the evaluation of the data density at points, i.e. new data. 

Let $z$ be a random variable distributed as $r(z)$ then a random variable $x = f(z|\theta)$, where f is a invertible and differentiable function with parameters $\theta$, is distributed as $q(x)$ with:
\begin{align*}
	q(x) = r(z)\left|\frac{\text{d} z}{\text{d} x}\right| = r(z)\left|\text{det}J_f(z)\right|^{-1}
\end{align*}
We call $r(z)$ the base distribution and $q(x)$ the target distribution.

To get a density estimation of new data $x'$ with some parameters we would vary $\theta$ until $q(x)$ is close to the target distribution of the data $p(x)$, then we can compute $r(f^{-1}(x'|\mathbf{\theta}))\left|\text{det}J_f(x'|\mathbf{\theta})\right|$ which is the estimate for the probability of the data. To do this we have to be able to compute the inverse transformation, its Jacobian determinant and evaluate the base distribution. As the base distribution we choose a multidimensional normal distribution. The dimension of this distribution is the number of features since the flow $f$ has to be injective.
\todo{citations}
\subsection{parameterize the transformation}
A list of the different parameterization implemented is in section .... Here we focus on the requirements that the transformation has to fulfill and how these can be achieved. 
different options -> section
how invertable (freezing or something else have to check), how neural nets, how expressive ( split in coupling layers).
figure of nf
\subsection{training the transformation}
In order to receive a distribution $q(x)$ that represents the target distribution $p(x)$ closely we have to vary the parameters $\theta$ of the transformation $f$. One can use gradient descent with a divergence as a loss function. In this implementation the Kullback-Leibler (KL) divergence is used as one of the most popular.

In this case where we have samples of the target distribution it is suitable to work with the forward KL divergence between $p(x)$ and $q(x|\theta)$:

\begin{align*}
	\mathcal{L}(\theta) &= D_{KL}\left[p(x)\middle\|q(x|\theta)\right]\\
	&=-\mathbb{E}_{p(x)}\left[\ln(q(x|\theta))\right] + \text{const} \\
	&\approx -\frac{1}{N}\sum_{n=1}^{N}\ln(r(f^{-1}(x_n|\theta))) + \ln\left|\text{det}J_{f^{-1}}(x_n|\theta)\right| + \text{const}
\end{align*} 

where the $x_n$ are the sampled target data. Thus we have to be able to compute $f^{-1}$, its jacobian determinant and evaluate $r(z)$ and since we want do use gradient descend we need to differentiate through them. \todo{citation}

\section{Normalizing flow categories}
list of different parametrizations -> all requirements in subsection
\todo{versch. NF erklären}


\section{Methods}
training testing val split

fixed base dist

tendorflow bijectors

how training (train utils)

adam optimizer

implementation mit tensorflow. citation LUKAS...
\todo{training erklären}
how checking (roc und hist der t und fraction of richtig class)
\todo{wie classifier bzw. was für auswertungen}



\section{Results}
loss function over time

hist of test stat der versch nf. 

plot der fraction of richtig class

vllt plot prob density

\section{Summary}

\section{Conclusion}

\bibliography{biblio.bibtex}
\todo{quellen raussuchen bzw. hier verweisen}
%
%\section{\label{sec:level1}First-level heading}
%
%This sample document was adapted from the template for papers
%in APS journals.
%It demonstrates proper use of REV\TeX~4.1 (and
%\LaTeXe) in mansucripts prepared for submission to APS
%journals. Further information can be found in the REV\TeX~4.1
%documentation included in the distribution or available at
%\url{http://authors.aps.org/revtex4/}.
%
%When commands are referred to in this example file, they are always
%shown with their required arguments, using normal \TeX{} format. In
%this format, \verb+#1+, \verb+#2+, etc. stand for required
%author-supplied arguments to commands. For example, in
%\verb+\section{#1}+ the \verb+#1+ stands for the title text of the
%author's section heading, and in \verb+\title{#1}+ the \verb+#1+
%stands for the title text of the paper.
%
%Line breaks in section headings at all levels can be introduced using
%\textbackslash\textbackslash. A blank input line tells \TeX\ that the
%paragraph has ended. Note that top-level section headings are
%automatically uppercased. If a specific letter or word should appear in
%lowercase instead, you must escape it using \verb+\lowercase{#1}+ as
%in the word ``via'' above.
%
%\subsection{\label{sec:level2}Second-level heading: Formatting}
%
%This file may be formatted in either the \texttt{preprint} or
%\texttt{reprint} style. \texttt{reprint} format mimics final journal output.
%Either format may be used for submission purposes. \texttt{letter} sized paper should
%be used when submitting to APS journals.
%
%\subsubsection{Wide text (A level-3 head)}
%The \texttt{widetext} environment will make the text the width of the
%full page, as on page~\pageref{eq:wideeq}. (Note the use the
%\verb+\pageref{#1}+ command to refer to the page number.)
%\paragraph{Note (Fourth-level head is run in)}
%The width-changing commands only take effect in two-column formatting.
%There is no effect if text is in a single column.
%
%\subsection{\label{sec:citeref}Citations and References}
%A citation in text uses the command \verb+\cite{#1}+ or
%\verb+\onlinecite{#1}+ and refers to an entry in the bibliography.
%An entry in the bibliography is a reference to another document.
%
%\subsubsection{Citations}
%Because REV\TeX\ uses the \verb+natbib+ package of Patrick Daly,
%the entire repertoire of commands in that package are available for your document;
%see the \verb+natbib+ documentation for further details. Please note that
%REV\TeX\ requires version 8.31a or later of \verb+natbib+.
%
%\paragraph{Syntax}
%The argument of \verb+\cite+ may be a single \emph{key},
%or may consist of a comma-separated list of keys.
%The citation \emph{key} may contain
%letters, numbers, the dash (-) character, or the period (.) character.
%New with natbib 8.3 is an extension to the syntax that allows for
%a star (*) form and two optional arguments on the citation key itself.
%The syntax of the \verb+\cite+ command is thus (informally stated)
%\begin{quotation}\flushleft\leftskip1em
%  \verb+\cite+ \verb+{+ \emph{key} \verb+}+, or\\
%  \verb+\cite+ \verb+{+ \emph{optarg+key} \verb+}+, or\\
%  \verb+\cite+ \verb+{+ \emph{optarg+key} \verb+,+ \emph{optarg+key}\ldots \verb+}+,
%\end{quotation}\noindent
%where \emph{optarg+key} signifies
%\begin{quotation}\flushleft\leftskip1em
%  \emph{key}, or\\
%  \texttt{*}\emph{key}, or\\
%  \texttt{[}\emph{pre}\texttt{]}\emph{key}, or\\
%  \texttt{[}\emph{pre}\texttt{]}\texttt{[}\emph{post}\texttt{]}\emph{key}, or even\\
%  \texttt{*}\texttt{[}\emph{pre}\texttt{]}\texttt{[}\emph{post}\texttt{]}\emph{key}.
%\end{quotation}\noindent
%where \emph{pre} and \emph{post} is whatever text you wish to place
%at the beginning and end, respectively, of the bibliographic reference
%(see Ref.~[\onlinecite{witten2001}] and the two under Ref.~[\onlinecite{feyn54}]).
%(Keep in mind that no automatic space or punctuation is applied.)
%It is highly recommended that you put the entire \emph{pre} or \emph{post} portion
%within its own set of braces, for example:
%\verb+\cite+ \verb+{+ \texttt{[} \verb+{+\emph{text}\verb+}+\texttt{]}\emph{key}\verb+}+.
%The extra set of braces will keep \LaTeX\ out of trouble if your \emph{text} contains the comma (,) character.
%
%The star (*) modifier to the \emph{key} signifies that the reference is to be
%merged with the previous reference into a single bibliographic entry,
%a common idiom in APS and AIP articles (see below, Ref.~[\onlinecite{epr}]).
%When references are merged in this way, they are separated by a semicolon instead of
%the period (full stop) that would otherwise appear.
%
%\paragraph{Eliding repeated information}
%When a reference is merged, some of its fields may be elided: for example,
%when the author matches that of the previous reference, it is omitted.
%If both author and journal match, both are omitted.
%If the journal matches, but the author does not, the journal is replaced by \emph{ibid.},
%as exemplified by Ref.~[\onlinecite{epr}].
%These rules embody common editorial practice in APS and AIP journals and will only
%be in effect if the markup features of the APS and AIP Bib\TeX\ styles is employed.
%
%\paragraph{The options of the cite command itself}
%Please note that optional arguments to the \emph{key} change the reference in the bibliography,
%not the citation in the body of the document.
%For the latter, use the optional arguments of the \verb+\cite+ command itself:
%\verb+\cite+ \texttt{*}\allowbreak
%\texttt{[}\emph{pre-cite}\texttt{]}\allowbreak
%\texttt{[}\emph{post-cite}\texttt{]}\allowbreak
%\verb+{+\emph{key-list}\verb+}+.

\end{document}
%
% ****** End of file templateForReport.tex ******
